\section{W3: Design models, RDDs}

\subsection{Static design model}
Static design model is a representation of software objects which define class names, attributes, and method signatures.\\
\textbf{Difference between domain class diagram and static design model:} the domain class diagram is a conceptual model of the domain, while the static design model is a model of the software objects. Domain models are used to influence the design models to reduce the representational gap between how stakeholders understand the entire work involved. Both models are designed in UML notation.
\textbf{Responsibility-driven design (RDD):} a design process that focuses on the responsibilities of the objects in the system. Assign responsibilities to classes, and then assign operations to classes.\\

\subsection{Dynamic design model}
Dynamic design model is a representation of how software objects interact with each other at runtime.\\
\textbf{Sequence diagram:} a diagram that shows the interactions between objects in a system, and the order of messages. The difference between a System Sequence Diagram and a Design Sequence Diagram is that the SSD shows the interactions between actors and the system, while the DSD shows the interactions between objects in the system.\\
